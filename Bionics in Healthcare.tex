\documentclass{report}
\usepackage{graphicx}
\usepackage[english]{babel}
\usepackage[utf8]{inputenc}

\begin{document}

\begin{centering}
\huge\bf{Term Project on Bionics in Healthcare}\\~\\

\includegraphics[scale=0.5]{National_Institute_of_Technology,_Raipur_Logo}\\~\\

\begin{LARGE}
\title\bf{ National Institute of Technology, Raipur}\\~\\
\end{LARGE}

\end{centering}
\subsection*\raggedright\large
\large{Submitted by:}\\~\\
\large Ranjeet Hansdah\\
\large\raggedright 21111045\\
\large\raggedright 1st Semester, Bio-Medical\\
\large\raggedright rh19102002@gmail.com\\~\\~\\

\raggedleft Under the supervision of:\\
Dr. Saurabh Gupta Sir,\\
Biomedical Department\\~\\
\newpage
\center
\begin{centering}
\LARGE\bf{Acknowledgement}
\end{centering}\\

\section*
\raggedright{I am grateful to Dr. Saurabh Gupta Sir, Biomedical Department for his proficient supervision of the term project on `Bionics in Healthcare'. I am very thankful to you sir for your guidance and support.}\\~\\~\\~\\
\raggedleft Ranjeet Hansdah\\
\raggedleft 21111045\\
\raggedleft 1st Semester, Biomedical\\
\raggedleft National Institute of Technology, Raipur\\
\newpage

\centering
\LARGE\bf {Abstract}

\section*\raggedright\normalfont\large {In this term project we are going to dicuss about the uses of bionics in health care. We will also see the working and applications of bionics in healthcare industry. And also learn the pros and cons of it and we will arrive at a proper conclusion.}
\raggedright
\section*{1)Introduction}
\includegraphics[scale=0.3]{Tilly+-+hand+in+hand}\\~\\
Bionics is a specific application of Bionics in engineering technology. It studies various special functions and characteristics of the biological mechanism, explore the ways and methods of its characteristic function imitation which can be used in engineering technology. Bionics describes all devices and prosthetics that replace a body part or organ to recover, improve or expand an individual’s functions. Most were aesthetic ‘replacements’ in the past, but nowadays we have bionic prosthetics and exoskeletons that combine advanced technologies to expand the mobility of the user. Bionics come in many shapes and sizes, including limbs, eyes, ears and organs. Different scientists are currently developing a fully bionic human who may function better than the original human. This is also known as human augmentation. Examples include implants that allow us to detect a magnetic field or electromagnetic radiation and brain-computer interfaces (BCIs) through which we can operate a prosthetic with nothing but our brain signals.
\section*{2)Elements of Bionics}
\subsection*{(1)Artificial Intelligence}
A completely established and a constantly evolving branch of computer science, Artificial Intelligence (AI) deals with intelligent behaviour in machines. Bionic Technology incorporates AI to expand the functionality of products that, are designed to mental and physical stress which an individual faces, following an amputation. This element also enhances the safety of the device along with the user. Thus, helps to stimulate voluntary control of the prosthetic limb.
\subsection*{(2)Power Motion}
The specialty of prosthetics has been known for many years largely because of its ability to replace the basic structural elements of the limbs after amputation, but now the ability to restore lost muscle function practically takes amputees to a higher level. From the subtle raising of the toe during the swing process to the powerful knee extension needed for ascending foot over foot stairs. Power movement provides unknown levels of flexibility.The most obvious advantage of devices incorporating power motion is reducing strain on the residual limb and protecting the body from the repercussions of muscular compensation.
\subsection*{(3)Sense-Think-Act}
The trio of functions of the device, which essentially includes:\\
i) Sophisticated sensors sample different aspects of real-time motion, often faster even than the human sensory system.\\
ii) Artificial intelligence continuously thinks through data from the sensors, analyzing and calculating the best move to make.\\
iii) Precision components act immediately to deliver the best possible function as instructed by the stream of signals sent from the artificial intelligence.\\
\section*{3)Some example of Major Bionics}
\subsection*{(1)Vision Bionics}
\includegraphics[scale=0.5]{twelve_bva}\\~\\
The bionic eye—or visual neuro prosthesis, as vision bionics are sometimes called—are bioelectronic implants that restore functional vision to people suffering from partial or total blindness. Researchers and device manufacturers who are designing bionic eyes confront two important challenges: the complexity of mimicking retinal function and the consumer preference (and constraint) for miniature devices that can be implanted into the eye. The way it works is that the camera, integrated into eyeglasses, captures images and transmits them to the portable processing unit, which wirelessly sends electrical signals to the implanted array. The array, in turn, converts these signals into electrical impulses that stimulate the retinal cells that connect to the optic nerve. One of the most prominent companies in this space is Second Sight Medical Products of Sylmar, Calif and second is Sight’s Argus II.
\subsection*{(2)Auditory Bionics}
\includegraphics[scale=0.6]{cochlear-implant-500x500}\\~\\
Cochlear implants, auditory brainstem implants and auditory midbrain implants are the three main classes of neuroprosthetic devices for people suffering from profound hearing loss. Auditory bionics create an artificial link between the source of sound and the brain—in this case, with a microelectronic array implanted either in the cochlea or the brain stem.Auditory bionics is more mature as a technology than vision bionics, with a larger innovation ecosystem, more commercial products, and greater adoption globally. The market is dominated by Cochlear Limited (Australia); Advanced Bionics (United States), a division of Sonova; MED-EL (Austria).
\subsection*{(3)Orthopedic Bionics}
\includegraphics[scale=0.8]{download}\\~\\
Orthopedic bionics are designed to restore motor functionality to the physically challenged.Bionic limbs are replacing prosthetic limbs, which were standard fare for more than 100 years. Despite notable innovations that resulted in lighter devices and better designs, prosthetic limbs did not provide the necessary functional restoration that bionic devices now do. A bionic limb is interfaced with a patient’s neuromuscular system for limb control—flexing, bending and grasping—using the brain. A similar functional pathway exists here: The damaged peripheral nerves are bypassed and a new electronic pathway connects the mechatronic limb with the brain. The well known companies who produces orthopedic bionics are Open Bionics and Touch Bionics (both UK-based start-ups), Martin Bionics (Oklahoma City, Okla.) and AlterG (Fremont, Calif.).
\section*{4)Applications and Benefits}
1. Bionics give back certain functions and indirectly offer more autonomy to the user. \\
2. Patients in the rehabilitation phase and users in the chronic phase can both benefit from bionics: the devices can serve to train and improve skills or to compensate for the definitive loss of capabilities .\\ 
3. Applications allow users to stay active, independent and healthy for longer while mitigating the negative impact of disabilities and amputations. \\
4. The user’s psychosocial condition is improved because they no longer have the feeling of ‘getting left behind’.
\section*{5)Advantanges of bionics}
1)Promising life-changing benefits of bionic limbs showed by long-term efficacy studies are compelling for patients.\\
2)It eliminates all typical burden associated with a socket, particularly the residuum’s skin problems.\\
3) It eases attachment and removal of the prosthesis. \\
4)It also provides a much more comfortable sitting position and allows a much larger range of movements.\\
5)There is overwhelming evidence that bionic limbs attached to osseointegrated implants significantly improve mobility. Users walk faster for longer bouts of daily and recreational activities.
\section*{6)Disadvantages of Bionics}
Common obstacles include:
1) Excessive sweating (hyperhidrosis), which can affect the fit of the prosthesis and lead to skin issues.\\
2) Changing residual limb shape.\\
3) Weakness in the residual limb, which may make it difficult to use the prosthesis for long periods of time.\\
4) Bionic limbs can potentially cause issues with implant stability, bone fracture, breakage of the implant parts and infection.\\
5) They also cost money paid either by the healthcare system or the users themselves as out-of-pocket expenses.
\section*{7)Future of Bionics}
\begin{itemize}
\item A relative new space in the bionics industry is robotic exoskeletons.These powered suits help patients who have limited or no muscle control walk, lift and generally be mobile. Exoskeletons are promising innovations that are expected to make a huge impact in the rehabilitation of patients who have suffered strokes or spinal injuries, and those who suffer from degenerative neuromuscular diseases such as amyotrophic lateral sclerosis.
\item Efforts must be made to lessen severity and incidence of adverse events of bone-anchored bionic prostheses.Prostheses must become safer and more accessible to a large population worldwide.
\item Safer treatment will also warrant a more successful development of the next generation of brain-activated neuroprostheses.
\item By computing physiological measures related to residuum health and creating a virtual replica of the residuum, the diagnostic device could improve care of patients and facilitate their rehabilitation.
\end{itemize}
\section*{8)Conclusion}
Bionics applications are integrating into our everyday lives more and more and their applications can make a major difference for individual users. Developments in bionics contribute to the evolution of the bionic human, though barriers remain in terms of technical challenges and ethical considerations.This term project gives a brief overview and analysis of the bionics and application, we can see the
development of bionics which can provide ways for engineering and technical innovation, believe that combination of bionics and engineering technology will have a broad application prospect. 
\section*{9)Refernces}
\begin{itemize}
\item https://zorgenablers.nl/en/bionics/
\item https://aabme.asme.org/posts/bionics-a-step-into-the-future
\item https://researchfeatures.com/future-bionic-limbs/
\end{itemize}



\end{document}